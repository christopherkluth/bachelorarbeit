%!TEX root = thesis.tex

\section{Schriftart, Zeichenkodierung und Sonderzeichen}

Folgende drei Befehle finden sich in so gut wie jedem deutschen LaTeX-Dokument, da sie den Umgang mit Sonderzeichen und Übersetzungen betreffen.
\subsection{Zeichenkodierung}

\textit{inputenc} (Einbindung mit \ybox{$\backslash$usepackage[utf8]\{inputenc\}}) ist für die Unterstützung erweiterter Eingabe-Zeichensätze mit ihren unterschiedlichen Kodierungen (z.B. ä, ö, ü usw.). Es dient der Umwandlung einer beliebigen Eingabe-Zeichenkodierung in eine interne LaTeX-Standardsprache. Neben utf8 und dessen Erweiterung utf8x gibt es noch andere Kodierungen, so zum Beispiel latin1 (wurde früher häufig unter Linux und Windows verwendet) oder applemac (eine Kodierung, die häufig für Textdateien unter Mac OS verwendet wurde). Wichtig ist, dass die Zeichenkodierung angegeben wird, die bei der Erstellung der TeX-Datei im Texteditor verwendet wurde. Im Austausch, bei wechselseitiger Bearbeitung, kann es bei unterschiedlicher Eingabe-Zeichenkodierung zu Problemen kommen.\cite{Ensenbach}
Schriftart

Die Standardschriftart von LaTeX ist die Computer-Modern-Schriftfamilie (CM). Standardmäßig bietet diese aber nicht alle 256 Zeichen des europäischen Zeichenvorrates in T1-Kodierung. Umlaute z. B. werden ohne den Einsatz der ec-Schriften nur aus den begrenzt vorhandenen Zeichen neu zusammengesetzt. Die ec- und tc-Schriften bieten dagegen einen Zeichenvorrat entsprechend der T1- und TS1-Kodierung. Eingebunden wird die T1-Version der Computer-Modern-Schriftsippe mit \ybox{$\backslash$usepackage[T1]\{fontenc\}}. Dadurch ist auch sichergestellt, dass in einem PDF Umlaute gefunden werden.

Die Computer-Modern-Schriftfamilie liegt in Metafont vor. Eine entsprechende Implementierung im PostScript-Type-1-Format sind die cm-super-Schriften.

In LaTeX finden sich 35 PostScript-Basisschriften, diese müssen ebenso wie andere Schriften gesondert eingebunden werden. Die sonst gängigen Type-1-Fonts müssen erst für TeX angepasst werden.

Für LaTeX stehen mehr als 100 freie Schriftarten zur Verfügung. Eine umfassende Übersicht mit Schrift-Beispielen bietet der LaTeX Font Catalogue.
Babel-System

Das Babel-System bietet eine Anpassung für LaTeX an viele Sprachen. Es wird seit 1992 von Johannes L. Braams und dem LaTeX-Team entwickelt und ist Bestandteil der populären TeX-Distribution TeX Live. Durch die Eingabe von \ybox{$\backslash$usepackage[ngerman]\{babel\}} kommt es zu einer Übersetzung der sprachspezifischen Befehle im Dokument („Inhaltsverzeichnis“ statt „table of contents“, „Kapitel“ anstelle von „chapter“ usw.). Für Benutzer der deutschen Version bietet das Babel-System beispielsweise die richtige Trennung nach der deutschen Rechtschreibung, die erleichterte Eingabe von Sonderzeichen, angepasste Typographie und Sprachumschaltung. Vorhergegangen war die Entwicklung von german.sty durch Hubert Partl und weitere Autoren.\cite{Babel}\cite{Kopka} Die deutsche LaTeX-Anpassung german.sty wurde zuvor beim sechsten TeX-Benutzertreffen im Oktober 1987 in Münster festgelegt.[14]