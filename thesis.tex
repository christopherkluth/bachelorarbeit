\documentclass{scrbook}

%!TEX root = thesis.tex

% Set german to default language and load english as well
\usepackage[english,ngerman]{babel}

% Set UTF8 as input encoding
\usepackage[utf8]{inputenc}

% Set T1 as font encoding
\usepackage[T1]{fontenc}
% Load a slightly more modern font
\usepackage{lmodern}
% Use the symbol collection textcomp, which is needed by listings.
\usepackage{textcomp}
% Load a better font for monospace.
\usepackage{courier}

% Set some options regarding the document layout. See KOMA guide
\KOMAoptions{%
  paper=a4,
  fontsize=12pt,
  parskip=half,
  headings=normal,
  BCOR=1cm,
  headsepline,
  DIV=12}

% do not align bottom of pages
\raggedbottom

% set style of captions
\setcapindent{0pt} % do not indent second line of captions
\setkomafont{caption}{\small}
\setkomafont{captionlabel}{\bfseries}
\setcapwidth[c]{0.9\textwidth}

% set the style of the bibliography
\bibliographystyle{alphadin}

% load extended tabulars used in the list of abbreviation
\usepackage{tabularx}

% load the color package and enable colored tables
\usepackage[dvipsnames,table]{xcolor}

% define new environment for zebra tables
\newcommand{\mainrowcolors}{\rowcolors{1}{maincolor!25}{maincolor!5}}
\newenvironment{zebratabular}{\mainrowcolors\begin{tabular}}{\end{tabular}}
\newcommand{\setrownumber}[1]{\global\rownum#1\relax}
\newcommand{\headerrow}{\rowcolor{maincolor!50}\setrownumber1}

% add main color to section headers
\addtokomafont{chapter}{\color{maincolor}}
\addtokomafont{section}{\color{maincolor}}
\addtokomafont{subsection}{\color{maincolor}}
\addtokomafont{subsubsection}{\color{maincolor}}
\addtokomafont{paragraph}{\color{maincolor}}

% do not print numbers next to each formula
\usepackage{mathtools}
\mathtoolsset{showonlyrefs}
% left align formulas
\makeatletter
\@fleqntrue\let\mathindent\@mathmargin \@mathmargin=\leftmargini
\makeatother

% Allow page breaks in align environments
\allowdisplaybreaks

% header and footer
\usepackage{scrpage2}
\pagestyle{scrheadings}
\setkomafont{pagenumber}{\normalfont\sffamily\color{maincolor}}
\setkomafont{pageheadfoot}{\normalfont\sffamily}
\setheadsepline{0.5pt}[\color{maincolor}]

% German guillemets quotes
\usepackage[german=guillemets]{csquotes}

% load TikZ to draw diagrams
\usepackage{tikz}

% load additional libraries for TikZ
\usetikzlibrary{%
  automata,%
  positioning,%
}

% set some default options for TikZ -- in this case for automata
\tikzset{
  every state/.style={
    draw=maincolor,
    thick,
    fill=maincolor!18,
    minimum size=0pt
  }
}

% load listings package to typeset sourcecode
\usepackage{listings}

% set some options for the listings package
\lstset{%
  upquote=true,%
  showstringspaces=false,%
  basicstyle=\ttfamily,%
  keywordstyle=\color{keywordcolor}\slshape,%
  commentstyle=\color{commentcolor}\itshape,%
  stringstyle=\color{stringcolor}}

% enable german umlauts in listings
\lstset{
  literate={ö}{{\"o}}1
           {Ö}{{\"O}}1
           {ä}{{\"a}}1
           {Ä}{{\"A}}1
           {ü}{{\"u}}1
           {Ü}{{\"U}}1
           {ß}{{\ss}}1
}

% define style for pseudo code
\lstdefinestyle{pseudo}{language={},%
  basicstyle=\normalfont,%
  morecomment=[l]{//},%
  morekeywords={for,to,while,do,if,then,else},%
  mathescape=true,%
  columns=fullflexible}

% load the AMS math library to typeset formulas
\usepackage{amsmath}
\usepackage{amsthm}
\usepackage{thmtools}
\usepackage{amssymb}

% load the paralist library to use compactitem and compactenum environment
\usepackage{paralist}

% load varioref and hyperref to create nicer references using vref
\usepackage[ngerman]{varioref}
\PassOptionsToPackage{hyphens}{url} % allow line break at hyphens in URLs
\usepackage{hyperref}

% setup hyperref
\hypersetup{breaklinks=true,
            pdfborder={0 0 0},
            ngerman,
            pdfhighlight={/N},
            pdfdisplaydoctitle=true}

% Fix bugs in some package, e.g. listings and hyperref
\usepackage{scrhack}

% define german names for referenced elements
% (vref automatically inserts these names in front of the references)
\labelformat{figure}{Abbildung\ #1}
\labelformat{table}{Tabelle\ #1}
\labelformat{appendix}{Anhang\ #1}
\labelformat{chapter}{Kapitel\ #1}
\labelformat{section}{Abschnitt\ #1}
\labelformat{subsection}{Unterabschnitt\ #1}
\labelformat{subsubsection}{Unterunterabschnitt\ #1}

% define theorem environments
\declaretheorem[numberwithin=chapter,style=plain]{Theorem}
\labelformat{Theorem}{Theorem\ #1}

\declaretheorem[sibling=Theorem,style=plain]{Lemma}
\labelformat{Lemma}{Lemma\ #1}

\declaretheorem[sibling=Theorem,style=plain]{Korollar}
\labelformat{Korollar}{Korollar\ #1}

\declaretheorem[sibling=Theorem,style=definition]{Definition}
\labelformat{Definition}{Definition\ #1}

\declaretheorem[sibling=Theorem,style=definition]{Beispiel}
\labelformat{Beispiel}{Beispiel\ #1}

\declaretheorem[sibling=Theorem,style=definition]{Bemerkung}
\labelformat{Bemerkung}{Bemerkung\ #1}

\input{macros}

% Set title and author used in the PDF meta data
\hypersetup{
  pdftitle={Wie schreibe ich eine Masterarbeit?},
  pdfauthor={Erika Mustermann}
}

% Depending on which of the following two color schemes you import your thesis will be in color or grayscale. I recommend to generate a colored version as a PDF and a grayscale version for printing.

%\input{schema-color}
\input{schema-gray}

\newcommand{\duedate}{15. Juli 2016}

\begin{document}
  \frontmatter
  %!TEX root = thesis.tex

\begin{titlepage}
	\thispagestyle{empty}
	
	\vskip1cm
	
	\pgfimage[height=2.5cm]{uni-logo-example\imagesuffix}
	
	\vskip2.5cm
	
	\LARGE
	
	\textbf{\sffamily\color{maincolor}\LaTeX}
	
	\textit{Makroprogramme für TeX}
	
	\normalfont\normalsize
	
	\vskip2em
	
	\textbf{\sffamily\color{maincolor}Bachelorarbeit}
	
	im Rahmen des Studiengangs \\
	\textbf{\sffamily\color{maincolor}TeXsatz für Dummies} \\
	der Universität zum Beispiel
	
	\vskip1em
	
	vorgelegt von \\
	\textbf{\sffamily\color{maincolor}Max Mustergruppe}
	
	\vskip1em
	
	ausgegeben und betreut von \\
	\textbf{\sffamily\color{maincolor}Prof. Dr. Mlte}
	
	\vskip1em
	
	mit Unterstützung von\\
	Wikipedia.org
	
	\vskip1em
	
	Die Arbeit ist im Rahmen einer Tätigkeit bei der Firma Wochenendarbeit GmbH entstanden.
	
	
	\vfill
	
	Internet, den \duedate
\end{titlepage}

  %!TEX root = thesis.tex

\cleardoublepage
\thispagestyle{plain}
\vspace*{\fill}

\section*{Erklärung}

Hiermit erkläre ich an Eides statt, dass ich die vorliegende
Arbeit ohne unzulässige Hilfe Dritter und ohne die Benutzung anderer
als der angegebenen Hilfsmittel selbständig verfasst habe;
die aus anderen Quellen direkt oder indirekt übernommenen Daten und Konzepte
sind unter Angabe des Literaturzitats gekennzeichnet.

\vskip2cm

\rule{5cm}{0.4pt}\\
(Max Mustergruppe)\\
Internet, den \duedate

  %!TEX root = thesis.tex

\cleardoublepage
\thispagestyle{plain}

\pdfbookmark{Kurzfassung}{kurzfassung}
\paragraph{Kurzfassung}
\textbf{LaTeX} (['la:t$\epsilon ç] $ oder ['la:t$\epsilon \chi ])$, in Eigenschreibweise \LaTeX\, ist ein Softwarepaket, das die Benutzung des Textsatzsystems TeX mit Hilfe von Makros vereinfacht. LaTeX liegt derzeit in der Version $2_\epsilon $ vor.

\cleardoublepage
\thispagestyle{plain}

\foreignlanguage{english}{%
\pdfbookmark{Abstract}{abstract}
\paragraph{Abstract} 
\textbf{LaTeX} $(/'lat\epsilon x/$ $\textbf{LAY}-tekh$, also pronounced as /'la:t$\epsilon k/$ $ \textbf{LAH}-tek$ or $/'leIt\epsilon k/$ $\textbf{LAY}-tek$, a shortening of Lamport TeX) is a document preparation system. When writing, the writer uses plain text as opposed to formatted text, as in WYSIWYG word processors like Microsoft Word or LibreOffice Writer. The writer uses markup tagging conventions to define the general structure of a document (such as article, book, and letter), to stylise text throughout a document (such as bold and italic), and to add citations and cross-references. A TeX distribution such as TeX Live or MikTeX is used to produce an output file (such as PDF or DVI) suitable for printing or digital distribution. Within the typesetting system, its name is stylised as LATEX.\\
LaTeX is widely used in academia for the communication and publication of scientific documents in many fields, including mathematics, physics, computer science, statistics, engineering, economics, philosophy, and political science. It also has a prominent role in the preparation and publication of books and articles that contain complex multilingual materials, such as Tamil, Sanskrit and Greek. LaTeX uses the TeX typesetting program for formatting its output, and is itself written in the TeX macro language.\\
LaTeX can be used as a standalone document preparation system or as an intermediate format. In the latter role, for example, it is sometimes used as part of a pipeline for translating DocBook and other XML-based formats to PDF. The typesetting system offers programmable desktop publishing features and extensive facilities for automating most aspects of typesetting and desktop publishing, including numbering and cross-referencing of tables and figures, chapter and section headings, the inclusion of graphics, page layout, indexing and bibliographies.\\
Like TeX, LaTeX started as a writing tool for mathematicians and computer scientists, but from early in its development it has also been taken up by scholars who needed to write documents that include complex math expressions or non-Latin scripts, such as Arabic, Sanskrit and Chinese.\\
LaTeX is intended to provide a high-level language that accesses the power of TeX in an easier way for writers. In short, TeX handles the layout side, while LaTeX handles the content side for document processing. LaTeX comprises a collection of TeX macros and a program to process LaTeX documents. Because the plain TeX formatting commands are elementary, it provides authors with ready-made commands for formatting and layout requirements such as chapter headings, footnotes, cross-references and bibliographies.\\
LaTeX was originally written in the early 1980s by Leslie Lamport at SRI International. The current version is LaTeX2e. LaTeX is free software and is distributed under the LaTeX Project Public License (LPPL).
}

  \cleardoublepage
  \phantomsection
  \pdfbookmark{Inhaltsverzeichnis}{tableofcontents}
  \markboth{Inhaltsverzeichnis}{}
  \tableofcontents

  \mainmatter
  \include{einleitung}
  \include{grundlagen}
  \include{konzept}
  \include{evaluation}
  \include{fazit}

  \appendix

  \include{appendix}

  \backmatter

  \cleardoublepage
  \phantomsection
  \pdfbookmark{Abbildungsverzeichnis}{listoffigures}
  \listoffigures

  \cleardoublepage
  \phantomsection
  \pdfbookmark{Tabellenverzeichnis}{listoftables}
  \listoftables

  \cleardoublepage
  \phantomsection
  \pdfbookmark{Definitions- und Theoremverzeichnis}{listoftheorems}
  \renewcommand{\listtheoremname}{Definitions- und Theoremverzeichnis}
  \listoftheorems[ignoreall,show={Lemma,Theorem,Korollar,Definition}]

  \include{abkuerzungen}

  \cleardoublepage
  \phantomsection
  \pdfbookmark{Literaturverzeichnis}{bibliography}
  \bibliography{literature}
\end{document}
