\documentclass{scrbook}

%!TEX root = thesis.tex

% Set german to default language and load english as well
\usepackage[english,ngerman]{babel}

% Set UTF8 as input encoding
\usepackage[utf8]{inputenc}

% Set T1 as font encoding
\usepackage[T1]{fontenc}
% Load a slightly more modern font
\usepackage{lmodern}
% Use the symbol collection textcomp, which is needed by listings.
\usepackage{textcomp}
% Load a better font for monospace.
\usepackage{courier}

% Set some options regarding the document layout. See KOMA guide
\KOMAoptions{%
  paper=a4,
  fontsize=12pt,
  parskip=half,
  headings=normal,
  BCOR=1cm,
  headsepline,
  DIV=12}

% do not align bottom of pages
\raggedbottom

% set style of captions
\setcapindent{0pt} % do not indent second line of captions
\setkomafont{caption}{\small}
\setkomafont{captionlabel}{\bfseries}
\setcapwidth[c]{0.9\textwidth}

% set the style of the bibliography
\bibliographystyle{alphadin}

% load extended tabulars used in the list of abbreviation
\usepackage{tabularx}

% load the color package and enable colored tables
\usepackage[dvipsnames,table]{xcolor}

% define new environment for zebra tables
\newcommand{\mainrowcolors}{\rowcolors{1}{maincolor!25}{maincolor!5}}
\newenvironment{zebratabular}{\mainrowcolors\begin{tabular}}{\end{tabular}}
\newcommand{\setrownumber}[1]{\global\rownum#1\relax}
\newcommand{\headerrow}{\rowcolor{maincolor!50}\setrownumber1}

% add main color to section headers
\addtokomafont{chapter}{\color{maincolor}}
\addtokomafont{section}{\color{maincolor}}
\addtokomafont{subsection}{\color{maincolor}}
\addtokomafont{subsubsection}{\color{maincolor}}
\addtokomafont{paragraph}{\color{maincolor}}

% do not print numbers next to each formula
\usepackage{mathtools}
\mathtoolsset{showonlyrefs}
% left align formulas
\makeatletter
\@fleqntrue\let\mathindent\@mathmargin \@mathmargin=\leftmargini
\makeatother

% Allow page breaks in align environments
\allowdisplaybreaks

% header and footer
\usepackage{scrpage2}
\pagestyle{scrheadings}
\setkomafont{pagenumber}{\normalfont\sffamily\color{maincolor}}
\setkomafont{pageheadfoot}{\normalfont\sffamily}
\setheadsepline{0.5pt}[\color{maincolor}]

% German guillemets quotes
\usepackage[german=guillemets]{csquotes}

% load TikZ to draw diagrams
\usepackage{tikz}

% load additional libraries for TikZ
\usetikzlibrary{%
  automata,%
  positioning,%
}

% set some default options for TikZ -- in this case for automata
\tikzset{
  every state/.style={
    draw=maincolor,
    thick,
    fill=maincolor!18,
    minimum size=0pt
  }
}

% load listings package to typeset sourcecode
\usepackage{listings}

% set some options for the listings package
\lstset{%
  upquote=true,%
  showstringspaces=false,%
  basicstyle=\ttfamily,%
  keywordstyle=\color{keywordcolor}\slshape,%
  commentstyle=\color{commentcolor}\itshape,%
  stringstyle=\color{stringcolor}}

% enable german umlauts in listings
\lstset{
  literate={ö}{{\"o}}1
           {Ö}{{\"O}}1
           {ä}{{\"a}}1
           {Ä}{{\"A}}1
           {ü}{{\"u}}1
           {Ü}{{\"U}}1
           {ß}{{\ss}}1
}

% define style for pseudo code
\lstdefinestyle{pseudo}{language={},%
  basicstyle=\normalfont,%
  morecomment=[l]{//},%
  morekeywords={for,to,while,do,if,then,else},%
  mathescape=true,%
  columns=fullflexible}

% load the AMS math library to typeset formulas
\usepackage{amsmath}
\usepackage{amsthm}
\usepackage{thmtools}
\usepackage{amssymb}

% load the paralist library to use compactitem and compactenum environment
\usepackage{paralist}

% load varioref and hyperref to create nicer references using vref
\usepackage[ngerman]{varioref}
\PassOptionsToPackage{hyphens}{url} % allow line break at hyphens in URLs
\usepackage{hyperref}

% setup hyperref
\hypersetup{breaklinks=true,
            pdfborder={0 0 0},
            ngerman,
            pdfhighlight={/N},
            pdfdisplaydoctitle=true}

% Fix bugs in some package, e.g. listings and hyperref
\usepackage{scrhack}

% define german names for referenced elements
% (vref automatically inserts these names in front of the references)
\labelformat{figure}{Abbildung\ #1}
\labelformat{table}{Tabelle\ #1}
\labelformat{appendix}{Anhang\ #1}
\labelformat{chapter}{Kapitel\ #1}
\labelformat{section}{Abschnitt\ #1}
\labelformat{subsection}{Unterabschnitt\ #1}
\labelformat{subsubsection}{Unterunterabschnitt\ #1}

% define theorem environments
\declaretheorem[numberwithin=chapter,style=plain]{Theorem}
\labelformat{Theorem}{Theorem\ #1}

\declaretheorem[sibling=Theorem,style=plain]{Lemma}
\labelformat{Lemma}{Lemma\ #1}

\declaretheorem[sibling=Theorem,style=plain]{Korollar}
\labelformat{Korollar}{Korollar\ #1}

\declaretheorem[sibling=Theorem,style=definition]{Definition}
\labelformat{Definition}{Definition\ #1}

\declaretheorem[sibling=Theorem,style=definition]{Beispiel}
\labelformat{Beispiel}{Beispiel\ #1}

\declaretheorem[sibling=Theorem,style=definition]{Bemerkung}
\labelformat{Bemerkung}{Bemerkung\ #1}

\input{macros}

% Set title and author used in the PDF meta data
\hypersetup{
  pdftitle={Wie schreibe ich eine Masterarbeit?},
  pdfauthor={Erika Mustermann}
}

% Depending on which of the following two color schemes you import your thesis will be in color or grayscale. I recommend to generate a colored version as a PDF and a grayscale version for printing.

%\input{schema-color}
\input{schema-gray}

\newcommand{\duedate}{15. Juli 2016}

\begin{document}
  \frontmatter
  %!TEX root = thesis.tex

\begin{titlepage}
	\thispagestyle{empty}
	
	\vskip1cm
	
	\pgfimage[height=2.5cm]{uni-logo-example\imagesuffix}
	
	\vskip2.5cm
	
	\LARGE
	
	\textbf{\sffamily\color{maincolor}\LaTeX}
	
	\textit{Makroprogramme für TeX}
	
	\normalfont\normalsize
	
	\vskip2em
	
	\textbf{\sffamily\color{maincolor}Bachelorarbeit}
	
	im Rahmen des Studiengangs \\
	\textbf{\sffamily\color{maincolor}TeXsatz für Dummies} \\
	der Universität zum Beispiel
	
	\vskip1em
	
	vorgelegt von \\
	\textbf{\sffamily\color{maincolor}Max Mustergruppe}
	
	\vskip1em
	
	ausgegeben und betreut von \\
	\textbf{\sffamily\color{maincolor}Prof. Dr. Mlte}
	
	\vskip1em
	
	mit Unterstützung von\\
	Wikipedia.org
	
	\vskip1em
	
	Die Arbeit ist im Rahmen einer Tätigkeit bei der Firma Wochenendarbeit GmbH entstanden.
	
	
	\vfill
	
	Internet, den \duedate
\end{titlepage}

  %!TEX root = thesis.tex

\cleardoublepage
\thispagestyle{plain}
\vspace*{\fill}

\section*{Erklärung}

Hiermit erkläre ich an Eides statt, dass ich die vorliegende
Arbeit ohne unzulässige Hilfe Dritter und ohne die Benutzung anderer
als der angegebenen Hilfsmittel selbständig verfasst habe;
die aus anderen Quellen direkt oder indirekt übernommenen Daten und Konzepte
sind unter Angabe des Literaturzitats gekennzeichnet.

\vskip2cm

\rule{5cm}{0.4pt}\\
(Max Mustergruppe)\\
Internet, den \duedate

  %!TEX root = thesis.tex

\cleardoublepage
\thispagestyle{plain}

\pdfbookmark{Kurzfassung}{kurzfassung}
\paragraph{Kurzfassung}
\textbf{LaTeX} (['la:t$\epsilon ç] $ oder ['la:t$\epsilon \chi ])$, in Eigenschreibweise \LaTeX\, ist ein Softwarepaket, das die Benutzung des Textsatzsystems TeX mit Hilfe von Makros vereinfacht. LaTeX liegt derzeit in der Version $2_\epsilon $ vor.

\cleardoublepage
\thispagestyle{plain}

\foreignlanguage{english}{%
\pdfbookmark{Abstract}{abstract}
\paragraph{Abstract} 
\textbf{LaTeX} $(/'lat\epsilon x/$ $\textbf{LAY}-tekh$, also pronounced as /'la:t$\epsilon k/$ $ \textbf{LAH}-tek$ or $/'leIt\epsilon k/$ $\textbf{LAY}-tek$, a shortening of Lamport TeX) is a document preparation system. When writing, the writer uses plain text as opposed to formatted text, as in WYSIWYG word processors like Microsoft Word or LibreOffice Writer. The writer uses markup tagging conventions to define the general structure of a document (such as article, book, and letter), to stylise text throughout a document (such as bold and italic), and to add citations and cross-references. A TeX distribution such as TeX Live or MikTeX is used to produce an output file (such as PDF or DVI) suitable for printing or digital distribution. Within the typesetting system, its name is stylised as LATEX.\\
LaTeX is widely used in academia for the communication and publication of scientific documents in many fields, including mathematics, physics, computer science, statistics, engineering, economics, philosophy, and political science. It also has a prominent role in the preparation and publication of books and articles that contain complex multilingual materials, such as Tamil, Sanskrit and Greek. LaTeX uses the TeX typesetting program for formatting its output, and is itself written in the TeX macro language.\\
LaTeX can be used as a standalone document preparation system or as an intermediate format. In the latter role, for example, it is sometimes used as part of a pipeline for translating DocBook and other XML-based formats to PDF. The typesetting system offers programmable desktop publishing features and extensive facilities for automating most aspects of typesetting and desktop publishing, including numbering and cross-referencing of tables and figures, chapter and section headings, the inclusion of graphics, page layout, indexing and bibliographies.\\
Like TeX, LaTeX started as a writing tool for mathematicians and computer scientists, but from early in its development it has also been taken up by scholars who needed to write documents that include complex math expressions or non-Latin scripts, such as Arabic, Sanskrit and Chinese.\\
LaTeX is intended to provide a high-level language that accesses the power of TeX in an easier way for writers. In short, TeX handles the layout side, while LaTeX handles the content side for document processing. LaTeX comprises a collection of TeX macros and a program to process LaTeX documents. Because the plain TeX formatting commands are elementary, it provides authors with ready-made commands for formatting and layout requirements such as chapter headings, footnotes, cross-references and bibliographies.\\
LaTeX was originally written in the early 1980s by Leslie Lamport at SRI International. The current version is LaTeX2e. LaTeX is free software and is distributed under the LaTeX Project Public License (LPPL).
}

  \cleardoublepage
  \phantomsection
  \pdfbookmark{Inhaltsverzeichnis}{tableofcontents}
  \markboth{Inhaltsverzeichnis}{}
  \tableofcontents

  \mainmatter
  %!TEX root = thesis.tex
\chapter{\LaTeX}
\section{Geschichte}

Das Basis-Programm von LaTeX ist TeX und wurde von Donald E. Knuth während seiner Zeit als Informatik-Professor an der Stanford University entwickelt. Auf TeX aufbauend entwickelte Leslie Lamport Anfang der 1980er Jahre\cite{Lamport} LaTeX, eine Sammlung von TeX-Makros, die die Benutzung für den durchschnittlichen Anwender gegenüber TeX vereinfachten und erweiterten. Der Name LaTeX ist eine Abkürzung für Lamport TeX.

Lamports Entwicklung von LaTeX endete gegen 1990 mit der Version 2.09.\cite{readme} Die aktuelle Version \LaTeX 2$\epsilon$ wurde ab 1989 von einer größeren Zahl von Autoren um Frank Mittelbach, Chris Rowley und Rainer Schöpf entwickelt.\cite{Mittelbach2010} Wesentliche Erweiterungen von Lamports Versionen bestanden in einem „vierdimensionalen“ Mechanismus für das Umschalten zwischen Zeichensätzen („New Font Selection Scheme“, „NFSS“), in einem komplexeren Mechanismus für das Einlesen von Zusatzpaketen und in einer entsprechenden „Paketschreiberschnittstelle“ für die Erstellung und Dokumentation auf LaTeX aufbauender Makropakete.\cite{usrguide} Die beiden Entwicklungsstufen von LaTeX sind nicht kompatibel.

\LaTeX 2$\epsilon$ ist seit Mitte der 1990er Jahre die am weitesten verbreitete Methode, TeX zu verwenden.
  %!TEX root = thesis.tex

\chapter{Grundprinzip}
\label{chapter-basics}

\section{Kein WYSIWYG}

Im Gegensatz zu anderen Textverarbeitungsprogrammen, die nach dem What-you-see-is-what-you-get-Prinzip funktionieren, arbeitet der Autor mit Textdateien, in denen er innerhalb eines Textes anders zu formatierende Passagen oder Überschriften mit Befehlen textuell auszeichnet. Das Beispiel unten zeigt den Quellcode eines einfachen LaTeX-Dokuments. Bevor das LaTeX-System den Text entsprechend setzen kann, muss es den Quellcode verarbeiten. Das dabei von LaTeX generierte Layout gilt als sehr sauber, sein Formelsatz als sehr ausgereift. Außerdem ist die Ausgabe u. a. nach PDF, HTML und PostScript möglich. LaTeX eignet sich insbesondere für umfangreiche Arbeiten wie Diplomarbeiten und Dissertationen, die oftmals strengen typographischen Ansprüchen genügen müssen. Insbesondere in der Mathematik und den Naturwissenschaften erleichtert LaTeX das Anfertigen von Schriftstücken durch seine komfortablen Möglichkeiten der Formelsetzung gegenüber üblichen Textverarbeitungssystemen. Das Verfahren von LaTeX wird auch mit WYSIWYAF (What you see is what you asked for.) umschrieben.

Das schrittweise Arbeiten erfordert vordergründig im Vergleich zu herkömmlichen Textverarbeitungen einerseits eine längere Einarbeitungszeit, andererseits kann das Aussehen des Resultats genau festgelegt werden. Die längere Einarbeitungszeit kann sich jedoch, insbesondere bei Folgeprojekten mit vergleichbarem Umfang oder ähnlichen Erfordernissen, lohnen.\cite{Fenn} Inzwischen gibt es auch grafische Editoren, die mit LaTeX arbeiten können und WYSIWYG oder WYSIWYM (What you see is what you mean.) bieten und ungeübten Usern den Einstieg deutlich erleichtern können. Beispiele für LaTeX-Entwicklungsumgebungen sind im Abschnitt Entwicklungsumgebungen aufgelistet.

\section{Logisches Markup}

Bei der Benutzung von LaTeX wird ein sogenanntes logisches Markup verwendet. Soll beispielsweise in einem Dokument eine Überschrift erstellt werden, so wird der Text nicht wie in TeX rein optisch hervorgehoben (etwa durch Fettdruck mit größerer Schrift, also: \colorbox{black!8}{$\backslash$font $\backslash$meinfont=cmb10 at 24pt $\backslash$meinfont Einleitung}), sondern die Überschrift wird als solche gekennzeichnet (z. B. mittels \colorbox{black!8}{$\backslash$section\{Einleitung\}}). In den Klassen- oder sty-Dateien wird global festgelegt, wie eine derartige Abschnittsüberschrift zu gestalten ist: „das Ganze fett setzen; mit einer Nummer davor, die hochzuzählen ist; den Eintrag in das Inhaltsverzeichnis vorbereiten“ usw. Dadurch erhalten alle diese Textstellen eine einheitliche Formatierung. Außerdem wird es dadurch möglich, automatisch aus allen Überschriften im Dokument mit dem Befehl \colorbox{black!8}{$\backslash$tableofcontents} ein Inhaltsverzeichnis zu generieren.

\section{Rechnerunabhängikeit}

Wie TeX selbst ist LaTeX weitestgehend rechnerunabhängig verwendbar. Das bedeutet, dass es für die meisten Betriebssysteme analog zu TeX auch für LaTeX lauffähige, produktiv einsetzbare LaTeX-Installationen gibt. Zu diesen Betriebssystemen gehören zum Beispiel Microsoft Windows von der Version 3.x bis zur aktuellen Version 10, Apple (Mac) OS X sowie diverse Linux-Distributionen. Unter der Voraussetzung, dass alle verwendeten Zusatzpakete (siehe unten) in geeigneten Versionen installiert sind, besteht der Vorteil der Verwendung von LaTeX darin, dass das Ergebnis unabhängig von der verwendeten Rechnerplattform und dem verwendeten Drucker in den beiden Ausgabeformaten DVI und PDF hinsichtlich der Schriftarten und -größen sowie der Zeilen- und Seitenumbrüche im Druck stets exakt gleich ist. LaTeX ist hierbei nicht auf die Schriftarten des jeweiligen Betriebssystems angewiesen. Jene Betriebssystem-Schriftarten sind häufig für die Anzeige am Monitor optimiert. LaTeX enthält eine Reihe eigener Schriftarten, die ihrerseits für den Druck optimiert sind.

\section{Technik, Alternativen, Entwicklung, Lizenz}
LaTeX bildet ein „Format“ in einem TeX-spezifischen Sinne. Es besteht im Kern aus einer Datei latex.ltx von Makrodefinitionen, die in ein Speicherabbild latex.fmt nach Ausführung der Definitionen umgewandelt wird. Diese Datei wird für jeden Dokumenterzeugungslauf zuerst von der „Engine“ (einem Binärprogramm wie TeX oder pdfTeX) eingelesen. Andere bedeutende Formate sind plain TeX und ConTeXt. Im Vergleich zu diesen existieren für LaTeX besonders viele Zusatzpakete – weitere Makrosammlungen – die auf den Makros in latex.ltx aufbauen und die zunächst mit der Engine und den Makros in latex.ltx gegebenen Darstellungsmöglichkeiten beträchtlich erweitern. Sie sind zu einem großen Teil von LaTeX-Anwendern zunächst für spezielle eigene Bedürfnisse (Veröffentlichungen, Lehrmaterial, Abschlussarbeiten) entwickelt und dann der Allgemeinheit mit meist freier Lizenz zur Verfügung gestellt worden. Speziell für diesen Zweck wurde die LaTeX Project Public License entwickelt, die auch für latex.ltx und weitere Pakete der Hauptentwickler von LaTeX gilt. Viele solcher Zusatzpakete sind auch im Auftrag von Verlagen, Fachzeitschriften, Fachgesellschaften und Hochschulen (für Abschlussarbeiten) für die Umsetzung ihrer Gestaltungsrichtlinien entwickelt worden.

  \include{konzept}
  \include{evaluation}
  \include{fazit}

  \appendix

  \include{appendix}

  \backmatter

  \cleardoublepage
  \phantomsection
  \pdfbookmark{Abbildungsverzeichnis}{listoffigures}
  \listoffigures

  \cleardoublepage
  \phantomsection
  \pdfbookmark{Tabellenverzeichnis}{listoftables}
  \listoftables

  \cleardoublepage
  \phantomsection
  \pdfbookmark{Definitions- und Theoremverzeichnis}{listoftheorems}
  \renewcommand{\listtheoremname}{Definitions- und Theoremverzeichnis}
  \listoftheorems[ignoreall,show={Lemma,Theorem,Korollar,Definition}]

  \include{abkuerzungen}

  \cleardoublepage
  \phantomsection
  \pdfbookmark{Literaturverzeichnis}{bibliography}
  \bibliography{literature}
\end{document}
