%!TEX root = thesis.tex
\chapter{\LaTeX}
\section{Geschichte}

Das Basis-Programm von LaTeX ist TeX und wurde von Donald E. Knuth während seiner Zeit als Informatik-Professor an der Stanford University entwickelt. Auf TeX aufbauend entwickelte Leslie Lamport Anfang der 1980er Jahre\cite{Lamport} LaTeX, eine Sammlung von TeX-Makros, die die Benutzung für den durchschnittlichen Anwender gegenüber TeX vereinfachten und erweiterten. Der Name LaTeX ist eine Abkürzung für Lamport TeX.

Lamports Entwicklung von LaTeX endete gegen 1990 mit der Version 2.09.\cite{readme} Die aktuelle Version \LaTeX 2$\epsilon$ wurde ab 1989 von einer größeren Zahl von Autoren um Frank Mittelbach, Chris Rowley und Rainer Schöpf entwickelt.\cite{Mittelbach2010} Wesentliche Erweiterungen von Lamports Versionen bestanden in einem „vierdimensionalen“ Mechanismus für das Umschalten zwischen Zeichensätzen („New Font Selection Scheme“, „NFSS“), in einem komplexeren Mechanismus für das Einlesen von Zusatzpaketen und in einer entsprechenden „Paketschreiberschnittstelle“ für die Erstellung und Dokumentation auf LaTeX aufbauender Makropakete.\cite{usrguide} Die beiden Entwicklungsstufen von LaTeX sind nicht kompatibel.

\LaTeX 2$\epsilon$ ist seit Mitte der 1990er Jahre die am weitesten verbreitete Methode, TeX zu verwenden.