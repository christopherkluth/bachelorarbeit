%!TEX root = thesis.tex

\section{Aussprache und Schreibweise von „\LaTeX“}

LaTeX wird im Englischen in der Regel als ['la:t$\epsilon$k] oder ['le$\tau$t$\epsilon$k] ausgesprochen (also nicht wie x [ks]).

Die Zeichen T, E, X im Namen kommen von den griechischen Großbuchstaben Tau, Epsilon und Chi, so wie sich auch der Name von TeX aus dem griechischen $\tau\epsilon\chi\nu\eta$ (Geschicklichkeit, Kunst, Technik) ableitet. Aus diesem Grund bestimmte TeX-Erfinder Donald E. Knuth die Aussprache als ['la:t$\epsilon$k],\cite{Knuth} das heißt mit einem stimmlosen velaren Frikativ wie im Neugriechischen, der ähnlich dem letzten Laut im deutschen Wort „Bach“ klingt.

So wird es auch im Deutschen als ['la:t$\epsilon\c{c}$] oder ['la:t$\epsilon \chi$] ausgesprochen. Dagegen äußerte Leslie Lamport, er schreibe keine bestimmte Aussprache für LaTeX vor.

Der Name wird traditionell mit einem speziellen typografischen Logo gedruckt (dargestellt in der Infobox am Anfang des Artikels). Kann im laufenden Text dieses Logo nicht genau reproduziert werden, wird das Wort in Unterscheidung zu Latex typischerweise in einer Kombination aus Groß- und Kleinbuchstaben als LaTeX dargestellt.

Die TeX-, LaTeX-[20] und XeTeX[21]-Logos können für den Einsatz in grafischen Web-Browsern nach den Vorgaben des LaTeX-internen Makros \LaTeX (oder \LaTeXe für die aktuelle Version) mit reinem CSS und XHTML dargestellt werden.[22]