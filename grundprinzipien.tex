%!TEX root = thesis.tex

\section{Grundprinzip}
\label{chapter-basics}

\subsection{Kein WYSIWYG}

Im Gegensatz zu anderen Textverarbeitungsprogrammen, die nach dem What-you-see-is-what-you-get-Prinzip funktionieren, arbeitet der Autor mit Textdateien, in denen er innerhalb eines Textes anders zu formatierende Passagen oder Überschriften mit Befehlen textuell auszeichnet. Das Beispiel unten zeigt den Quellcode eines einfachen LaTeX-Dokuments. Bevor das LaTeX-System den Text entsprechend setzen kann, muss es den Quellcode verarbeiten. Das dabei von LaTeX generierte Layout gilt als sehr sauber, sein Formelsatz als sehr ausgereift. Außerdem ist die Ausgabe u. a. nach PDF, HTML und PostScript möglich. LaTeX eignet sich insbesondere für umfangreiche Arbeiten wie Diplomarbeiten und Dissertationen, die oftmals strengen typographischen Ansprüchen genügen müssen. Insbesondere in der Mathematik und den Naturwissenschaften erleichtert LaTeX das Anfertigen von Schriftstücken durch seine komfortablen Möglichkeiten der Formelsetzung gegenüber üblichen Textverarbeitungssystemen. Das Verfahren von LaTeX wird auch mit WYSIWYAF (What you see is what you asked for.) umschrieben.

Das schrittweise Arbeiten erfordert vordergründig im Vergleich zu herkömmlichen Textverarbeitungen einerseits eine längere Einarbeitungszeit, andererseits kann das Aussehen des Resultats genau festgelegt werden. Die längere Einarbeitungszeit kann sich jedoch, insbesondere bei Folgeprojekten mit vergleichbarem Umfang oder ähnlichen Erfordernissen, lohnen.\cite{Fenn} Inzwischen gibt es auch grafische Editoren, die mit LaTeX arbeiten können und WYSIWYG oder WYSIWYM (What you see is what you mean.) bieten und ungeübten Usern den Einstieg deutlich erleichtern können. Beispiele für LaTeX-Entwicklungsumgebungen sind im Abschnitt Entwicklungsumgebungen aufgelistet.

\subsection{Logisches Markup}

Bei der Benutzung von LaTeX wird ein sogenanntes logisches Markup verwendet. Soll beispielsweise in einem Dokument eine Überschrift erstellt werden, so wird der Text nicht wie in TeX rein optisch hervorgehoben (etwa durch Fettdruck mit größerer Schrift, also: \ybox{\textbackslash font \textbackslash meinfont=cmb10 at 24pt \textbackslash meinfont Einleitung}), sondern die Überschrift wird als solche gekennzeichnet (z. B. mittels \ybox{\textbackslash section\{Einleitung\}}). In den Klassen- oder sty-Dateien wird global festgelegt, wie eine derartige Abschnittsüberschrift zu gestalten ist: „das Ganze fett setzen; mit einer Nummer davor, die hochzuzählen ist; den Eintrag in das Inhaltsverzeichnis vorbereiten“ usw. Dadurch erhalten alle diese Textstellen eine einheitliche Formatierung. Außerdem wird es dadurch möglich, automatisch aus allen Überschriften im Dokument mit dem Befehl \ybox{\textbackslash tableofcontents} ein Inhaltsverzeichnis zu generieren.

\subsection{Rechnerunabhängikeit}

Wie TeX selbst ist LaTeX weitestgehend rechnerunabhängig verwendbar. Das bedeutet, dass es für die meisten Betriebssysteme analog zu TeX auch für LaTeX lauffähige, produktiv einsetzbare LaTeX-Installationen gibt. Zu diesen Betriebssystemen gehören zum Beispiel Microsoft Windows von der Version 3.x bis zur aktuellen Version 10, Apple (Mac) OS X sowie diverse Linux-Distributionen. Unter der Voraussetzung, dass alle verwendeten Zusatzpakete (siehe unten) in geeigneten Versionen installiert sind, besteht der Vorteil der Verwendung von LaTeX darin, dass das Ergebnis unabhängig von der verwendeten Rechnerplattform und dem verwendeten Drucker in den beiden Ausgabeformaten DVI und PDF hinsichtlich der Schriftarten und -größen sowie der Zeilen- und Seitenumbrüche im Druck stets exakt gleich ist. LaTeX ist hierbei nicht auf die Schriftarten des jeweiligen Betriebssystems angewiesen. Jene Betriebssystem-Schriftarten sind häufig für die Anzeige am Monitor optimiert. LaTeX enthält eine Reihe eigener Schriftarten, die ihrerseits für den Druck optimiert sind.

\subsection{Technik, Alternativen, Entwicklung, Lizenz}
LaTeX bildet ein „Format“ in einem TeX-spezifischen Sinne. Es besteht im Kern aus einer Datei latex.ltx von Makrodefinitionen, die in ein Speicherabbild latex.fmt nach Ausführung der Definitionen umgewandelt wird. Diese Datei wird für jeden Dokumenterzeugungslauf zuerst von der „Engine“ (einem Binärprogramm wie TeX oder pdfTeX) eingelesen. Andere bedeutende Formate sind plain TeX und ConTeXt. Im Vergleich zu diesen existieren für LaTeX besonders viele Zusatzpakete – weitere Makrosammlungen – die auf den Makros in latex.ltx aufbauen und die zunächst mit der Engine und den Makros in latex.ltx gegebenen Darstellungsmöglichkeiten beträchtlich erweitern. Sie sind zu einem großen Teil von LaTeX-Anwendern zunächst für spezielle eigene Bedürfnisse (Veröffentlichungen, Lehrmaterial, Abschlussarbeiten) entwickelt und dann der Allgemeinheit mit meist freier Lizenz zur Verfügung gestellt worden. Speziell für diesen Zweck wurde die LaTeX Project Public License entwickelt, die auch für latex.ltx und weitere Pakete der Hauptentwickler von LaTeX gilt. Viele solcher Zusatzpakete sind auch im Auftrag von Verlagen, Fachzeitschriften, Fachgesellschaften und Hochschulen (für Abschlussarbeiten) für die Umsetzung ihrer Gestaltungsrichtlinien entwickelt worden.
