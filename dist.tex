\section{LaTeX-Software (Distributionen)}

Um mit LaTeX zu arbeiten, werden grundlegend zwei Dinge benötigt: Einerseits die (im Hintergrund arbeitende) LaTeX-Software und andererseits eine Eingabe-/Steuerungssoftware, mit der der LaTeX-Code eingegeben und die Ausführung angestoßen wird (siehe hierzu Abschnitt „Entwicklungsumgebung“).

Die LaTeX-Software besteht aus den TeX/LaTeX-Programmen, Schriften, Skripten und Zusatzprogrammen. Der einfachste Weg, um die LaTeX-Software zu installieren ist, eine Distribution zu wählen. Diese installiert alle notwendigen Programme und die gebräuchlichsten Zusätze. Bekannte Distributionen sind proTeXt (für Windows, basiert auf MiKTeX), TeX Live (Unix/Linux/Windows/Mac), MacTeX (für Mac OS X, basierend auf TeX Live).